\capitulo{2}{Objetivos del proyecto}

% Este apartado explica de forma precisa y concisa cuales son los objetivos que se persiguen con la realización del proyecto. Se puede distinguir entre los objetivos marcados por los requisitos del software a construir y los objetivos de carácter técnico que plantea a la hora de llevar a la práctica el proyecto.

%Descripción inicial del objetivo del proyecto
%Este proyecto es una prueba tecnologíca para mejorar la eficiencia de la ejecución de un tipo de algoritmos como se indica en Big Data.

Este trabajo ha surgido como una prueba destinada a observar el rendimiento de algoritmos de \textit{Big Data} en un entorno paralelo, así como a la implementación de diferentes algoritmos que permitan la selección de las mejores instancias durante la fase de pre procesamiento de la minería de datos.

\section{Estudio del rendimiento de la minería de datos en un modelo de ejecución en paralelo.}

La minería de datos se trasladó a los entornos paralelos como una manera de poder tratar con grandes conjuntos de datos.

Como primera aproximación, utilizaremos dos herramientas diferentes, Weka(ver sección \nameref{sec:DefWeka}) para la ejecución lineal y Spark(ver sección \nameref{sec:DefSpark}) para la ejecución en paralelo, para comparar el rendimiento de ambas modelos de ejecución frente a conjuntos de datos de diferentes proporciones.

%===============================================
%A AÑADIR MÁS ADELANTE
%===============================================
Probablemente se añadan pruebas con distribuciones de computación en la nube.

\section{Implementación de algoritmos de selección de instancias.}

Se programarán un conjunto de algoritmos que puedan aplicarse sobre grandes conjuntos de instancias con el fin de reducir dicho conjunto y mejorar el rendimiento de cualquier otro algoritmo que use posteriormente los datos. Para una definición más precisa de lo que es un algoritmo de selección de instancias ver la sección \nameref{sec:DefAlgSel}

Se ha realizado la implementación de los siguientes algoritmos:

\begin{itemize}
	\item Algoritmos implementados.
\end{itemize}

%===============================================
%A AÑADIR MÁS ADELANTE
%===============================================
Probablemente los algoritmos se usen también en las distribuciones de Spark que encontremos por la nube o se realicen mediciones de rendimiento.