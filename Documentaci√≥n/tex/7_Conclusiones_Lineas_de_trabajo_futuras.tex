\capitulo{7}{Conclusiones y Líneas de trabajo futuras}

%Todo proyecto debe incluir las conclusiones que se derivan de su desarrollo. Éstas pueden ser de diferente índole, dependiendo de la tipología del proyecto, pero normalmente van a estar presentes un conjunto de conclusiones relacionadas con los resultados del proyecto y un conjunto de conclusiones técnicas. 
%Además, resulta muy útil realizar un informe crítico indicando cómo se puede mejorar el proyecto, o cómo se puede continuar trabajando en la línea del proyecto realizado. 

La realización del proyecto ha permitido comprobar la complejidad de la programación paralela y aplicada a escenarios donde el rendimiento, la comunicación entre unidades de procesamiento o la falta de recursos son aspectos a tener muy en cuenta.

Los objetivos propuestos al comienzo del trabajo se han cumplido, logrando la implementación paralela de varios algoritmos de selección de instancias y su posterior comparación con las implementaciones anteriores. Igualmente, otras metas definidas \textit{a posteriori}, como la implementación de una interfaz gráfica y la ejecución en la nube, han sido correctamente cumplidas.

Finalmente, y centrándonos en el ámbito de las comparaciones, este proyecto nos han ayudado a desmentir la hipótesis inicial con la que se comenzó a trabajar: que todos los algoritmos propuestos conseguirían reducir el tiempo de ejecución con respecto a sus antiguas versiones. Se ha visto como el algoritmo LSHIS, debido a su rapidez, es difícil de mejorar cuando intentamos paralelizarlo. Por el contrario, el algoritmo DemoIS muestra resultados muy favorables incluso con conjuntos de instancias pequeños.


\subsection{Lineas de trabajo futuras}
La propia naturaleza del proyecto ofrece una gran variedad de alternativas para seguir trabajando:

\begin{itemize}
\item \textbf{Continuar trabajando sobre los algoritmos ya implementados:} Aunque es posible que el algoritmo LSHIS pueda ser abordado utilizando un paradigma distinto (\textit{streaming}), DemoIS todavía puede ofrecer un gran margen de mejora. Una implementación paralela del algoritmo K-NN o una nueva estrategia para la creación de subconjuntos podrían suponer, no solo una mejora en el comportamiento del algoritmo, sino también una todavía más rápida ejecución. Además, existen algunas variaciones de los algoritmos que no han sido implementadas en su versión paralela.

\item \textbf{Continuar las mediciones en la nube:} Aunque se han realizado numerosos experimentos para comprobar el rendimiento de los algoritmos propuestos, la mayoría de ellos se han realizado en una única máquina, limitando nuestros recursos o el tamaño de los conjuntos de datos a utilizar. Es cierto que se han realizado algunas pruebas en otro tipo de entornos, pero se han visto limitadas por problemas técnicos. Realizar más mediciones en un auténtico clúster permitiría realizar comparaciones con conjuntos de datos mucho más grandes, bastantes más recursos, y probablemente ayudase a definir más claramente el alcance del paradigma de programación en paralelo.

\item \textbf{Mejorar métodos de lectura/escritura:} Un aspecto que no ha sido tratado durante este proyecto ha sido la manera en la que leer o almacenar los conjuntos de datos. Técnicas como la lectura de datos de un sistema de ficheros distribuido o de una base de datos podrían ser de mucha utilidad en este área y, por lo tanto, podrían ser abordados en futuras mejoras.

\item \textbf{Seguir ampliando el número de algoritmos de minería de datos:} La estructura del proyecto y el campo en el que se desarrolla, dejan la puerta abierta a la implementación de multitud de algoritmos que se ejecuten en paralelo, pues no existe actualmente una gran librería que cubra este ámbito.


\end{itemize}



