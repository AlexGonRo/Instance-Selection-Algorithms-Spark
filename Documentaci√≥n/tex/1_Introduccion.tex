\capitulo{1}{Introducción}\label{chap:Introduccion}

%Descripción del contenido del trabajo y del estrucutra de la memoria y del resto de materiales entregados.

La minería de datos es un área que se ha mantenido en intensa y constante evolución desde su aparición formal en los años 80 y 90. Desde el comienzo, todos estos cambios han tenido por objetivo buscar una solución a los problemas que la minería de datos iba planteando. En lo que se refiere a la etapa actual, uno de los problemas más importantes a los que se está haciendo frente es la gran cantidad de los datos y el creciente número de atributos de los mismos~\cite{DataMiningConcepts}, que hacen imposible seguir aplicando aproximaciones anteriores por problemas de eficiencia.

A lo largo de esta memoria vamos a trabajar sobre algunos de los paradigmas que han surgido en el área para poder hacer frente a grandes volúmenes de datos: la paralelización de las tareas de minería de datos y la preselección de instancias para reducir el tamaño inicial del conjunto de datos y mejorar su calidad. 

\subsection{Estructura de la memoria}
	La memoria ha mantenido la estructura definida en primera instancia por los profesores del tribunal del TFG:
	\begin{itemize}
	\item \textbf{Objetivos del proyecto:} Donde definiremos cuáles serán las metas que hemos intentado alcanzar con la realización del trabajo.
	\item \textbf{Conceptos teóricos:} Donde se darán a conocer todos aquellos conceptos que, sin estar incluidos dentro del conocimiento básico, son necesarios para la comprensión del proyecto.
	\item \textbf{Técnicas y herramientas:} Donde se explicarán las metodologías usadas para llevar a cabo el proyecto, así como cualquier herramienta que haya sido utilizada en la elaboración del mismo, incluyendo las causas de su elección en caso de considerarlo necesario.
	\item Se añadirán nuevas secciones. \todo{Faltan indicar nuevas secciones en la memoria.}
%	\item \textbf{Aspectos relevantes del desarrollo del proyecto:} Donde explicaremos todos aquellos apartados que consideremos de interés durante la evolución del proyecto.
%	\item \textbf{Trabajos relacionados:}Donde se dejará constancia de cualquier otro trabajo que se hubiese realizado sobre el área tratada en este proyecto.
%	\item \textbf{Conclusiones y líneas futuras de trabajo:} Donde se dejará constancia de todo lo aprendido o extraído del proyecto, así como de diferentes posibilidades para continuar trabajando.
	\end{itemize}

\subsection{Materiales entregados}
Junto con la memoria se hará entrega de los siguientes archivos:
	\begin{itemize}
%	\item \textbf{Máquina virtual:} Se dejará a disposición del tribunal una imagen virtual de Ubuntu 14.04 que contendrá todos los materiales necesarios para probar el funcionamiento del proyecto.
	\item \textbf{Anexo:} Documento adicional que contendrá el plan de proyecto,los requisitos diseño y los manuales de usuario y de programador.
	\item Se añadirán nuevos materiales. \todo{Faltan indicar nuevos materiales entregados en la memoria.}
	\end{itemize}