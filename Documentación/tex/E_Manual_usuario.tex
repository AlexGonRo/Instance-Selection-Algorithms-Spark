\apendice{Documentación de usuario}

\section{Introducción}

A lo largo de este apéndice vamos a indicar el material necesario para poder ejecutar el proyecto, así como la manera de instalarlo, y la forma de hacer funcionar el proyecto una vez contemos con todos los elementos requeridos.

\section{Requisitos de usuarios}

\todo{Completar }


\section{Instalación}\label{sec:Instalacion}

A continuación se van a definir los métodos de instalación de todos los componentes necesarios para ejecutar el proyecto, por si pudiera resultar de utilidad.

Estos métodos de instalación hacen referencia a la instalación en máquinas con un sistema operativo basado en alguna distribución de Debian. En nuestro caso, esta distribución ha sido Ubuntu 14.04.

\subsection{Oracle Java 8}

Aunque existen distribuciones libres de Java, usaremos la que proporciona Oracle (\url{http://www.java.com}) por el hecho de que incluye una serie de programas que pueden usarse para medir el rendimiento de aplicaciones Java (JConsole y JVisualVM). Cualquier otra distribución será igualmente válida, pero es posible que no incluya estas herramientas. Podemos acceder tanto al JRE como al JDK desde la siguiente dirección: \url{http://www.oracle.com/technetwork/java/javase/downloads/index.html}.

Sin embargo, Oracle no proporciona una instalación automática de Java para Linux. Es por esta razón vamos a utilizar un PPA (Personal Package Archive) que proporciona un instalador para diferentes versiones de Java (\url{https://launchpad.net/~webupd8team/+archive/ubuntu/java}). Este instalador no contiene ningún archivo Java, pero será el encargado de descargarlos en nuestra máquina (de igual manera que podríamos hacerlo manualmente) y realizar la instalación automáticamente, facilitando el proceso de instalación.

Una vez entendido esto, ejecutaremos los siguientes comandos:

\begin{lstlisting}
$ sudo add-apt-repository ppa:webupd8team/java
$ sudo apt-get update
$ sudo apt-get install oracle-java8-installer
\end{lstlisting}

Podemos comprobar si la instalación ha sido correcta ejecutando el comando \colorbox{lightgray}{\lstinline|java -version|} en la terminal. Si todo ha salido bien deberíamos recibir una salida similar a la siguiente:

\begin{lstlisting}
$ java -version
java version "1.8.0_66"
Java(TM) SE Runtime Environment (build 1.8.0_66-b17)
Java HotSpot(TM) 64-Bit Server VM (build 25.66-b17,
mixed mode)
\end{lstlisting}

Finalmente, podemos ejecutar el siguiente comando para configurar correctamente las variables del sistema que Java necesita:

\begin{lstlisting}
$ sudo apt-get install oracle-java8-set-default
\end{lstlisting}

\subsection{Scala 2.11.7}
Aunque podemos descargar la última versión de Scala desde la página oficial (\url{http://www.scala-lang.org/}), la descarga que se nos ofrece por defecto es un archivo .tar.gz que nos obligaría a realizar toda la instalación manualmente (suponiendo que estamos en un sistema Ubuntu como el utilizado durante el proyecto).

Por ello, vamos a realizar la instalación mediante un archivo .deb (paquete Debian) que también puede ser utilizado por nuestro sistema y que nos ahorrará realizar la instalación manualmente. Podemos acceder al repositorio que guarda este paquete desde un navegador (\url{http://www.scala-lang.org/files/archive/}) o descargarlo mediante el siguiente comando:

\begin{lstlisting}
$ sudo wget www.scala-lang.org/files/archive
 /scala-2.11.7.deb
\end{lstlisting}

Independientemente del método seguido, una vez tengamos el archivo en nuestro ordenador ejecutamos el siguiente comando desde la carpeta que contenga el paquete .deb:

\begin{lstlisting}
$ sudo dpkg -i scala-2.11.7.deb
\end{lstlisting}

Si todo ha salido correctamente, una vez termine de ejecutarse la orden anterior podemos ejecutar el comando \colorbox{lightgray}{\lstinline|scala -version|} y esperar una salida similar a esta:

\begin{lstlisting}
$ scala -version
Scala code runner version 2.11.7 -- Copyright 2002-2013
\end{lstlisting}

Recordar que, en el caso de utilizar cualquier otra versión de Scala, esta tiene que ser compatible con la versión Java que se encuentre instalada. Por ejemplo, Java 8 solo puede ser utilizado a partir de versiones 2.11 y será obligatorio a partir de la versión de Scala 2.12. \cite{Scala2.12Roadmap}


\subsection{Apache Maven}

Lo primero que debemos hacer es descargarnos el paquete que contiene la herramienta. Esto podemos hacerlo desde el navegador en la página oficial de Apache Maven (\url{https://maven.apache.org/}) o mediante el siguiente comando en consola:

\begin{lstlisting}
wget  http://apache.rediris.es/maven/maven-3/3.3.3
 /binaries/apache-maven-3.3.3-bin.tar.gz
\end{lstlisting}

Recordar que el código de arriba es orientativo, pudiéndose seleccionar otro lugar desde donde realizar la descarga u otra versión del producto.

Descomprimimos el paquete descargado y lo movemos  a la carpeta \colorbox{lightgray}{\lstinline|/usr/local|}:

\begin{lstlisting}
$ tar -zxf apache-maven-3.3.3-bin.tar.gz
$ sudo mv -R apache-maven-3.3.3 /usr/local
$ sudo ln -s /usr/local/apache-maven-3.3.3/bin/mvn 
  /usr/bin/mvn
\end{lstlisting}


Podemos comprobar que la instalación ha sido realizada correctamente si al escribir el comando \colorbox{lightgray}{\lstinline|mvn --version|} recibimos una salida parecida a la siguiente:

\begin{lstlisting}
$ mvn --version
Apache Maven 3.3.3 (7994120775791599e205a5524ec3e0dfe41d4a06;
 2015-04-22T13:57:37+02:00)
Maven home: /usr/local/apache-maven-3.3.3
Java version: 1.8.0_66, vendor: Oracle Corporation
Java home: /usr/lib/jvm/java-8-oracle/jre
Default locale: en_US, platform encoding: UTF-8
OS name: "linux", version: "3.19.0-25-generic", arch:
 "amd64", family: "unix"
\end{lstlisting}


\subsection{Apache Spark 1.5.1}

Nos dirigiremos a la página oficial de Apache Spark (\url{http://spark.apache.org/}). Elegiremos la versión 1.5.1 por ser la utilizada a lo largo de la práctica, y descargaremos el código fuente. Igualmente, y como hemos mencionado en otras instalaciones, podemos ejecutar el siguiente comando para hacernos con el paquete:

\begin{lstlisting}
$ wget http://apache.rediris.es/spark/spark-1.5.1
  /spark-1.5.1.tgz
\end{lstlisting}


Una vez tengamos el archivo en nuestra máquina lo descomprimimos y movemos a la carpeta que deseemos lanzando estos comandos desde el directorio que contenga el paquete descargado:

\begin{lstlisting}
$ tar -xvf spark-1.5.1.tgz
$ sudo mv -R spark-1.5.1 /opt
\end{lstlisting}

Finalmente vamos a construir Spark utilizando los siguientes comandos desde la carpeta donde lo hemos ubicado. En nuestro caso, estará en \colorbox{lightgray}{\lstinline|/opt/spark-1.5.1|}:

\begin{lstlisting}
$ sudo ./dev/change-scala-version.sh 2.11
$ sudo mvn -Pyarn -Phadoop-2.6 -Dscala-2.11 -Pnetlib-lgpl
  -DskipTests clean package
\end{lstlisting}

Es importante incluir la opción \colorbox{lightgray}{\lstinline|-Pnetlib-lgpl|} para que Spark incluya una serie de clases utilizadas por su librería MLlib\cite{SparkDependencies}.

\todo{Queda pendiente si al final vamos a necesitar -Pnetlib-lgpl o no, revisar esto cuando tengamos todo
hecho}

Esta última operación llevará un tiempo. 

Finalmente, si deseamos comprobar que Spark se ha desplegado correctamente podemos ejecutar, por ejemplo, el intérprete de comandos de Scala para Spark:

\begin{lstlisting}
$ ./bin/spark-shell
\end{lstlisting}

Aunque recibiremos una serie de avisos (\textit{warnings}) debido a que no hemos configurado ciertos aspectos de Spark, el intérprete debería poder lanzarse y funcionar sin problemas.


\todo{¿Hablar de los alias y variables de entorno? Vamos a dejar esto de momento}
%WEKA_HOME="/opt/weka-3-7-13/"
%SPARK_HOME="/opt/spark-1.5.1/"
%Añadido en el fichero /etc/environment

\subsection{Algoritmos de selección de instancias}

Para descargar el resultado final de nuestro proyecto

\todo{Completar cuando lo tengamos todo. Probablemente tengamos que indicar como instalar git}


\section{Instalaciones Opcionales}

\subsection{Weka}

No se trata de un programa necesario para la ejecución del proyecto, pero, dado que parte del mismo ha sido destinado a comparar los resultados entre Apache Spark y Weka, lo consideraremos como un programa a incluir en este apéndice.

Durante la realización de la práctica se han utilizado dos versiones diferentes de este software: Weka-3.6.13 y Weka-3.7.13. La instalación de ambos es similar, por lo que para no tener información duplicada en este anexo explicaremos la instalación refiriéndonos siempre a la versión 3.7.13

Lo primero que haremos será descargar Weka desde su página oficial (\url{http://www.cs.waikato.ac.nz/ml/weka/}). De entre las posibles opciones que nos ofrecen, hemos de seleccionar la que nos permite descargar un archivo de extensión .zip destinado a ``\textit{Other platforms (Linux, etc.)}''

Una vez tengamos el archivo, escribiremos en la terminal el siguiente comando, que nos permitirá descomprimir el directorio y colocarlo en la carpeta \colorbox{lightgray}{\lstinline|/opt|}:

\begin{lstlisting}
$ sudo unzip ~/Downloads/weka-3-7-13.zip -d /opt
\end{lstlisting}

Para comprobar el correcto funcionamiento de Weka podemos dirigirnos a la carpeta de instalación y ejecutar el siguiente comando. Si todo ha sido realizado correctamente Weka debería iniciarse:

\begin{lstlisting}
$ cd /opt/weka-3-7-13
$ java -jar weka.jar
\end{lstlisting}

\section{Manual del usuario}

\todo{Completar cuando sea definitiva la estructura de ficheros y, sobretodo, la salida del resultado}
