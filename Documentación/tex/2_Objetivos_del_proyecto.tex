\capitulo{2}{Objetivos del proyecto}

% Este apartado explica de forma precisa y concisa cuales son los objetivos que se persiguen con la realización del proyecto. Se puede distinguir entre los objetivos marcados por los requisitos del software a construir y los objetivos de carácter técnico que plantea a la hora de llevar a la práctica el proyecto.

En el escenario actual de la minería de datos (ver definición en \ref{DefMineria}) la creciente cantidad de datos que requieren ser analizados, así como el aumento de su complejidad,  ha generado un problema a la hora de tratar esos datos con eficiencia y velocidad. El hecho de que este problema parezca ir en aumento ha dado lugar a una búsqueda de soluciones que proporcionen una alta capacidad de cálculo y una fácil escalabilidad~\cite{DataMiningConcepts}. 

Una de las propuestas ha sido la posibilidad de ejecutar las labores de minería de manera paralela. Por otro lado, analizando el problema desde un enfoque diferente, aunque no incompatible, han adquirido gran popularidad los algoritmos encargados de reducir el número de instancias y atributos, de manera que podamos hacer más pequeño el conjunto a estudiar o incluso reducir su nivel de ruido. Este trabajo surge bajo la motivación de explorar estas nuevas soluciones. 

Será un proyecto con dos objetivos principales: por un lado la implementación de diferentes algoritmos de reducción de instancias que, además de correr en paralelo, nos permitirán mejorar la calidad de los datos y reducir su tamaño. Por otro, el análisis del rendimiento de implementaciones lineales frente a nuestras implementaciones paralelas.

Como objetivo adicional, surgido durante la realización del trabajo, tendremos en consideración la creación de un entorno visual y más intuitivo donde el usuario pueda ejecutar nuestro código de una manera más sencilla.

Igualmente, tendremos en consideración la posibilidad de ejecutar el resultado de este proyecto en una red de nodos real, al ser este el entorno final donde este programa está pensado para ser ejecutado.

\section{Implementación de algoritmos de selección de instancias}

Se programarán un conjunto de algoritmos que puedan aplicarse sobre grandes conjuntos de instancias con el fin de reducir su tamaño sin perjudicar en exceso la información que transmiten. Para una definición más precisa de lo que es un algoritmo de selección de instancias ver la sección \ref{sec:DefAlgSel}.

Se ha realizado la implementación de los siguientes algoritmos:

\begin{itemize}
	\item \textit{Localy sensitive hashing instance selection} (LSHIS) (ver \ref{sec:defLSHIS}) \cite{LSHISPaper}
	\item \textit{Democratic instance selection} (DemoIS) (ver \ref{sec:defDemoIS}) \cite{DemoISPaper}
\end{itemize}


\section{Estudio del rendimiento de la minería de datos en un modelo de ejecución en paralelo}\label{sec:estudio}

El principal objetivo de este estudio será comprobar los tiempos de ejecución según qué paradigma (secuencial o paralelo) y según que conjunto de datos.

Utilizaremos dos herramientas diferentes: Weka (ver definición en \ref{sec:DefWeka}) para la ejecución lineal y Spark (ver definición \ref{sec:DefSpark}) para la ejecución en paralelo.

Las pruebas serán realizadas en una única máquina local, con la posibilidad de extenderlas a un clúster si cumpliésemos con el objetivo definido en \ref{sec:ejecucionNube}.

\section{Ejecución del proyecto en un clúster en la nube}\label{sec:ejecucionNube}

A fin de asegurar que el trabajo podría funcionar sin ningún tipo de complicación en un entorno real, se ha propuesto como objetivo el despliegue de la aplicación en algún servicio que ofrezca computación en la nube.

Se ha elegido para ello el servicio Google Cloud Dataproc \cite{dataprocSoft}, que ofrece la posibilidad de realizar trabajos de Spark. Igualmente, se ha considerado la posibilidad de ejecutar lanzamientos en un servidor clúster ya contratado por la Universidad de Burgos donde ,además, podríamos llevar a cabo un estudio sobre la eficiencia de nuestra aplicación, tal y como se define en \ref{sec:estudio}.


\section{Implementación de una interfaz gráfica de usuario}

Con el objetivo de facilitar el uso de la aplicación y sus algoritmos, se ha incluido el desarrollo de una interfaz gráfica que permita plantear experimentos sin la necesidad de que el usuario tenga que acceder a su consola de comandos.

Dicha implementación no solo permitirá configurar de manera simple los algoritmos, sino que también será un entorno donde poder seleccionar algunas de las opciones de lanzamiento de Spark o incluso seleccionar los conjuntos de datos a utilizar.

